% !TEX root = 0.gesamt.tex
\section{Didaktische Analyse}

Die SuS setzen sich in der geplanten Stunde mit dem Zusammenhang zwischen
Wahlbeteiligung und Sozialstruktur auseinander und beurteilen die daraus
entstehenden Probleme für die repräsentative Demokratie. Dies geschieht anhand
der Bremer Bürgerschaftswahl 2015 sowie anlässlich der Bürgerschaftswahl 2019,
bei der viele der SuS zum ersten Mal wählen dürfen. Folgt man den Prinzipien
der didaktischen Analyse von \citeA{klafki1962}, ergibt sich hieraus
unmittelbar die \textbf{Gegenwartsbedeutung} der gesamten UE wie auch der
geplanten Stunde. Die Auseinandersetzung mit Funktion, Ablauf, Inhalten und
Problemen von Wahlen als einem wesentlichen Pfeiler der repräsentativen
Demokratie ist für die SuS unmittelbar bedeutsam und fördert ihre
\textbf{politische Handlungsfähigkeit}. Diese Bedeutung wurde auch deutlich
anhand der vielen Fragen der SuS anlässlich des Einstiegs, die für den weiteren
Verlauf der UE die Grundlage bilden. Auch die Auseinandersetzung mit dem Thema
der geplanten Stunde ist für die SuS unmittelbar individuell bedeutsam, schon
alleine durch die ihnen bevorstehende prinzipielle Entscheidung, ob sie
überhaupt wählen gehen \textit{wollen}.\footnote{Bei einer in der
Einstiegsstunde anonym durchgeführten \enquote{Sonntagsfrage} (\emph{Was
würdest Du wählen, wenn am kommenden Sonntag Wahl wäre?}) entschieden sich
sechs von 25 SuS für Nichtwahl.} Im Rahmen ihrer Rolle als Bürger im System der
repräsentativen Demokratie werden diese Fragen auch in absehbarer Zukunft für
die SuS bedeutsam sein (\textbf{Zukunftsbedeutung}). Mit diesem
\emph{persönlichen Bezug} ist eine Ebene der \textbf{Struktur des Inhalts}
bereits skizziert. Eine weitere Ebene stellt die soziale Selektivität der
Wahlbeteiligung dar, mit der gleichzeitig die soziale Ungleichheit von
Stadtteilen in Bremen (und anderen Großstädten) thematisiert ist
(\textbf{politische Analysefähigkeit}). Die dritte Ebene thematisiert die Frage
nach der aus der \emph{sozialen Selektivität} resultierenden faktisch nicht
mehr vorhandenen \emph{sozialen Repräsentativität} von Wahlergebnissen und den
Konsequenzen (\textbf{politische Urteilsfähigkeit}). Das \textbf{Entwerfen} und
\textbf{Diskutieren} von Lösungsvorschlägen ist eine vierte Ebene der
inhaltlichen Struktur, die die SuS aber eigenständig schriftlich bearbeiten
(Hausaufgabe).

Die Fokussierung auf die lokalen Bremer Wahlen ist zum einen lebensweltlich
begründet (Lebensumfeld der SuS), zum anderen werden die bundesweiten Probleme
der sinkenden Wahlbeteiligung, der sozialen Selektivität und den daraus
resultierenden Fragen nach Repräsentativität und der demokratischen
Legitimation von Regierung und Parlament hier, wie die Sachanalyse gezeigt hat,
besonders deutlich. Es handelt sich um ein generelles Problem des Systems der
repräsentativen Demokratie in Deutschland, welches sich anhand der Bremer
Bürgerschaftswahlen und Stadtteilstruktur beispielhaft erschließen lässt
(\textbf{exemplarische Bedeutung}). Die \textbf{Zugänglichkeit} des Themas wird
dabei zum einen über den persönlichen Bezug (ausgehend von der Reflexion des
eigenen Wahlverhaltens), zum anderen über den lokalen Bezug zum eigenen
Lebensumfeld (Bremen, eigenes Stadtviertel) ermöglicht.
