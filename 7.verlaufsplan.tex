% !TEX root = 0.gesamt.tex

 % \newcolumntype{L}[1]{>{\raggedright\arraybackslash}p{#1}} % linksbündig mit Breitenangabe

 \begin{landscape}
 %\thispagestyle{empty}
 \section{Verlaufsplan}

 \singlespacing
 \scriptsize
 \begin{tabular}{p{2cm}|p{4.5cm}|p{4.5cm}|p{1.5cm}|p{2cm}|p{4.5cm}}

\textbf{Phase} &
\textbf{Aktivität Lehrkraft} &
\textbf{Aktivität SuS} &
\textbf{Sozialform} &
\textbf{Medien, Material} &
\textbf{didaktisch-methodischer Kommentar}

\\
\hline
% Begrüßung??
\textbf{Einstieg}
&
- begrüßt die SuS stellt die Kommission vor

- präsentiert am Smartboard die Zahl 50,2

- Frage: \emph{was könnte diese Zahl im Kontext der UE bezeichnen?}

- nennt und notiert das Thema der Stunde an der Tafel
&
- begrüßen L und die Kommission

- nennen mögliche Bedeutungen der Zahl
&
Plenum / UG
&
Smartboard, PC, Tafel
&
- Fokussierung, Neugier

- bei Bedarf werden Hilfestellungen eingeblendet

\begin{comment}
\begin{itemize*}
    \item bereitet vor und mit längerem text, aber das kostet zuviel platz

    \item kontrolliert
\end{itemize*}
\end{comment}

\\
\hline

\textbf{Problemati"-sierung I}
&
Leitfrage: \emph{Warum ist die geringe Wahlbeteiligung überhaupt ein Problem?}

- Sammelt Antworten stichpunktartig an der Tafel

- Präsentation des \enquote{wahren} Wahlergebnisses

- Einführung und Visualisierung des Begriffs \emph{Repräsentationsquote}
&
äußern Vermutungen und begründete Hypothesen
&
UG
&
Smartboard, PC, Tafel
&
- Problematisierung auf Grundlage der Ergebnisse aus der vorhergehenden Stunde

- (Fach)-Sprachkompetenz

\\
\hline

\textbf{Erarbeitung}
&
- teilt die Gruppen ein und verteilt Arbeitsblätter

- zeigt das vorbereitete Padlet und erläutert die Aufgabenstellung
&
- bearbeiten die Aufgabe

- Schriftführer notieren die Ergebnisse im Padlet
&
GA
&
Smartboard, PC, Arbeitsblätter, Smartphones, Padlet
&
leistungshetero"-gene Gruppeneinteilung (Diff"-erenzierung)

\\
\hline

\textbf{Austausch, Sicherung}
&
- ergänzt und korrigiert ggf.

- notiert zusammenfassendes Endergebnis im Padlet

- führt den Begriff \emph{soziale Selektivität} (und ergänzend den Begriff {\emph{prekäre Wahlen}) ein und notiert ihn als Oberbegriff im Padlet
&
- Gruppensprecher*innen präsentieren und erläutern die Ergebnisse

- fragen nach
&
UG, Schülerpräsentation
&
Smartboard, Padlet
&

\\
\hline

\textbf{Problemati"-sierung II}
&
Frage: \emph{Was bedeutet die soziale Selektivität für die Repräsentationsquote?}
&
- diskutieren und erörtern

- formulieren den Zusammenhang und die daraus resultierenden Probleme von sozialer Selektivität und Repräsentation
&
UG, Plenum
&
&
Vertiefung von Problematisierung I

\\
\hline


\textbf{Hausaufgabe}
&
erläutert, zeigt die Hausaufgabe auf itslearning
&
entwerfen Lösungsvorschläge vor dem Hintergrund der in der Stunde aufgezeigten Problematik und recherchieren ggf. dazu.
&
EA
&
&
- Abgabe erfolgt über itslearning

- Diskussion der Lösungsvorschläge in nachfolgender Stunde

\\
\hline

\textbf{Didaktische Reserve}
&
bitte um Stellungnahme zu dem Vorschlag, eine Wahlpflicht einzuführen
&
- nehmen Stellung und beurteilen eine mögliche Wahlpflicht
&
Plenum
&
&
Vorgriff auf Hausaufgabe

\end{tabular}

 \normalsize
 \onehalfspacing
 \end{landscape}
