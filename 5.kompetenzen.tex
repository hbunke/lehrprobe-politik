% !TEX root = 0.gesamt.tex

\section{Kompetenzen}

% muss hier eigentlich nicht rein?  \subsection{Leitidee der Unterrichtsstunde}

% \subsection{Kompetenzraster}

\scriptsize
\begin{singlespacing}

\begin{longtable}{p{1.5cm}|p{2cm}|p{4cm}|p{4cm}|p{3cm}}

\hline
\textbf{Kompetenz Dimension} &

\textbf{Allgemeine fachspezifische Kompetenzbereiche} &

\textbf{Standards der Bildungspläne \newline \newline die SuS können...} &

\textbf{Aufgabenstruktur (Per"-formanz) \newline \newline ...indem sie...} &

\textbf{Differenzierte Kompetenz"-niveaus}

\\
\hline

\textbf{Fach"-kompetenz} &
Politische Analysefähig"-keit

Politische Urteilsfähigkeit

Politische Handlungsfähigkeit

&
...aktuelle politische Prozesse auf demokratische Kernprinzipien analysieren und gesellschaftliche Machtverhältnisse und Interessensgegensätze reflektieren...,

...das gesellschaftliche System im Hinblick auf soziale Strukturen an ausgewählten Beispielen beschreiben und erklären...,

...grundlegende gesellschaftliche Herausforderungen unter dem Gesichtspunkt sozialer Gerechtigkeit benennen und erklären sowie an ausgewählten Beispielen gesellschaftliche Entwicklungen beschreiben und die damit zusammenhängenden Probleme benennen...,
&

...die sich aus der niedrigen Wahlbeteiligung ergebenden Probleme \textbf{benennen} und \textbf{beschreiben}

...in Arbeitsgruppen aus den gegebenen Teilstatistiken den Zusammenhang zwischen Wahlbeteiligung und Sozialstruktur \textbf{herausarbeiten}

...die niedrige Wahlbeteiligung unter dem Aspekt der sozialen Selektivität und der Repräsentationsquote des Parlaments \textbf{beurteilen}

...erste eigene Lösungsvorschläge für das Problem der geringen Wahlbeteiligung \textbf{entwerfen} (Hausaufgabe)
&
- die Statistiken eigenständig analysieren \newline
- die Statistiken mit Hilfestellung analysieren \newline


\\
\hline
\textbf{Methoden"-kompetenz} &

Erschließung von Informationen

Anwendung erschlossener Informationen

Problemlösungs- und
Argumentationsfähigkeit

&

...Informationen aus unterschiedlichen Quellen entnehmen, problemangemessen auswerten und in Zusammenhänge einordnen,

...Arbeitsergebnisse angemessen (u.a. mit Hilfe
moderner Medien) präsentieren.


&

...in Arbeitsgruppen aus den gegebenen Teilstatistiken den Zusammenhang zwischen Wahlbeteiligung und Sozialstruktur \textbf{herausarbeiten}

... die Arbeitsergebnisse in einem Satz schriftlich \textbf{zusammenfassen}, der Klasse \textbf{präsentieren} und \textbf{erläutern}

& % hier noch Differenzierung?
- diskutieren und formulieren (Gruppe) \newline
- präsentieren und erläutern (Referent*in der Gruppe)

\\
\hline
\textbf{Sozial"-kompetenz} &
Teamfähigkeit

Kommunikations"-fähigkeit

& % Bildungspläne leer, da steht nix über Sozialkompetenz :-)

&
...in der Gruppe gemein"-sam einen statistischen Sachverhalt erarbeiten und gemeinsam formulieren

& % hier noch Differenzierung?
- verbal diskutieren \newline
- ausschließlich schriftliche Formulierung

\\
\hline

\textbf{Personal"-kompetenz} &
Logisches Denken

Lernbereitschaft
&  % keine personalen Kompetenzen in den Bildungsplänen
&
...aus den gegebenen Teilstatistiken den Zusammenhang zwischen Wahlbeteiligung und Sozialstruktur \textbf{herausarbeiten}

...sie Interesse an zahlreichen, auch statistischen, neuen Informationen zeigen und sich dieses Wissen schnell aneignen
&

\\
\hline
\textbf{Sprach"-kompetenz}
& % keine allgemeinen Kompetenzen
& % keine Bildungsstandards
&
... Fachbegriffe präzise einsetzen können

... einen aus statistischen Materialien erarbeiteten Zusammenhang in einem Satz formulieren können

&

\\
\hline

\end{longtable}



\end{singlespacing}
\normalsize
