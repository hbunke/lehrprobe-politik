% !TEX root = 0.gesamt.tex
\section{Sachanalyse}

An den letzten Bürgerschaftswahlen im Land Bremen im Jahr 2015 beteiligten sich
lediglich 50,2\% der Wahlberechtigten, d.h. von knapp 488.000 Wahlberechtigten
nahmen nur gut 244.000  \cite<vgl. das amtliche Endergebnis in>[S.
7]{wayand2015} dieses Recht in Anspruch. Damit setzte sich ein seit 1987 in
Bremen zu verzeichnender Trend sinkender Wahlbeteiligung fort \cite[S.
10]{wayand2015}. Bei keiner Bürgerschaftswahl seit 1946 war die Wahlbeteiligung
niedriger. Besonders drastisch fiel dabei auch das nochmalige Absinken um 5,3
Prozentpunkte gegenüber der vorhergehenden Wahl von 2011 aus \cite[S.
11]{wayand2015}. Dabei ist der Trend sinkender Wahlbeteiligung zwar ein
bundesweiter, allerdings fällt er in Bremen besonders stark aus. In keinem
westlichen Bundesland gab es seit 1946 eine niedrigere Wahlbeteiligung bei
Landtagswahlen \cite[S. 7]{vehrkamp2015}.

Mit der Wahlbeteiligung sinkt auch die \emph{Repräsentationsquote} drastisch.
Der aktuelle rot-grüne Senat repräsentiert nur noch rund 23\% der
Wahlberechtigten. Rechnet man die nicht wahlberechtigten Ausländer hinzu, sinkt
die Repräsentationsquote des Senats sogar auf unter 20\%, die der Bürgerschaft
(Landtag) auf 41,6\%.

Schon diese geringe Repräsentationsquote ist ein prinzipielles Problem für eine
repräsentative Demokratie. \enquote{Hinter der regierenden Mehrheit steht dann
unter Umständen nur noch eine zahlenmäßig kleine Minderheit} \cite{decker2016},
was unmittelbar die Frage nach der demokratischen Legitimation von Parlament
und Regierung aufwirft. Diese Frage wird noch dringender, betrachtet man die
steigende \emph{soziale Selektivität} der Wahlbeteiligung \cite<vgl. hierzu
ausführlich>{schafer2015, decker2016, vehrkamp2015, schafer2013}. Hier
konstatiert die Forschung einen eindeutigen Befund. \enquote{Die soziale Lage
eines Ortsteils bestimmt die Höhe der Wahlbeteiligung: Je höher der Anteil von
Haushalten aus den sozial schwächeren Milieus, je höher die Arbeitslosigkeit,
je geringer der formale Bildungsstand und je geringer die durchschnittliche
Kaufkraft der Haushalte in einem Ortsteil, desto geringer ist die
Wahlbeteiligung} \cite[S. 9]{vehrkamp2015}. Dies gilt nicht nur für Bremen,
sondern für ganz Deutschland, besonders die Großstädte \cite{schafer2013,
schafer2015}. Hintergrund der geringen Wahlbeteiligung ist also die soziale
Spaltung der Wählerschaft. Wissenschaft \cite{vehrkamp2015, schafer2013} und
auch Politik\footnote{ Siehe z.B. das Interview mit dem Vorsitzenden der
SPD"~Fraktion in der Bremischen Bürgerschaft im Weser"~Kurier vom 30.01.2019
\cite{boekhoff2019}.} sprechen inzwischen von \emph{prekären Wahlen}. Es
entsteht ein selbsteskalierender Prozess: Wahlenthaltung führt zu mangelnder
Interessenvertretrung im Parlament, wodurch Nichtwähler wiederum noch weniger
Gründe haben, zur Wahl zu gehen \cite{decker2016}. Die sinkende
Repräsentationsquote betrifft also vor allem die sozial benachteiligten
Wählergruppen und Stadtteile und ist damit nicht nur ein prinzipielles
Legitimations-Problem, sondern auch und vor allem ein sozialstrukturelles.
