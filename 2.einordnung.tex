% !TEX root = 0.gesamt.tex

% background color for special row in unterrichtssequenz
\definecolor{Gray}{gray}{0.9}


\section{Einordnung des Themas in curriculare Vorgaben und die Unterrichtssequenz}

Die geplante Unterrichtsstunde ist Teil der UE \emph{Wahlen, Wähler und
Wahlkampf am  Beispiel der Bremer Bürgerschaftswahl 2019}, die sich einordnet
in den im Bremer Bildungsplan für Politik in der 10. Klasse genannten
Themenbereich \enquote{Gesellschaftliche Realität(en)} und die dort
aufgeführten Inhalte \enquote{Demokratie als Gesellschaftsprinzip und
Gesellschaftliche Kräfteverhältnisse} und \enquote{Sozialstruktur,
demografische Entwicklungen und deren Auswirkungen} \cite[S.
35]{dersenatorfurbildungundwissenschaft2006}. Im schulinternen Curriculum wird
unter dem o.g. Themenbereich explizit das in der UE behandelte Thema
\enquote{Wahlen, Wähler und Wahlkampf} konkretisiert. Innerhalb der Unterrichtseinheit folgt das
behandelte Thema \emph{Wahlbeteiligung} auf zwei eher wissensorientierte
Stunden (Wahlamt, \enquote{Wissen wie Wählen}), die ausgingen von konkreten
Fragen der SuS, sowie einer eher theoretischen Auseinandersetzung mit der
Funktion von Wahlen in Demokratien. Im zweiten Teil der UE folgt die
Auseinandersetzung mit Inhalten, Parteiprogrammen und Wahlkampfthemen der
Bremer Bürgerschaftswahl sowie zum Abschluss die Analyse des Wahlergebnisses.


\subsection{Unterrichtssequenz}

\begin{footnotesize}

\begin{tabular}{|p{2cm}|p{12.75cm}|}
\hline
\textbf{Stunde} & \textbf{Inhalt}  \\
\hline
1 & Einstieg und Themenerarbeitung \\
\hline
2 & Exkursion Wahlamt \\
\hline
3 & WWW - Wissen Wie Wählen \\
\hline
4 & Wahlen und Demokratie \\
\hline
\rowcolor{Gray}
\textbf{5} & \textbf{Wahlbeteiligung als sozialstrukturelles Problem} \\
\hline
6 & Parteien und Wahlprogramme \\
\hline
7 & Wahlkampf: Themen, Analysen, Prognosen \\
\hline
8 & Wahlanalyse und UE Auswertung\\
\hline

\end{tabular}

\end{footnotesize}

\vspace{0.5cm}
