% !TEX root = 0.gesamt.tex

\section{Methodische Überlegungen}

Basierend auf den didaktischen Überlegungen wird zunächst mit dem
\textbf{Einstieg} versucht, ein Problem zu benennen und ein allgemeines
\emph{Problembewusstsein} zu schaffen \cite[S. 427]{greving2014}. Dies
geschieht zunächst mit der Neugier und frageweckenden \cite{meyer1990}
Beamer-Projektion der \enquote{reinen} (kein \% Zeichen, kein weiterer Text)
Zahl der Beteiligungsquote der Bürgerschaftswahl von 2015 (50,2) an das
Smartboard, verbunden mit der Frage: Was könnte diese Zahl in unserem Kontext
bezeichnen? Nach der Klärung wird die Zahl von mir kurz ergänzt mit dem Verlauf
der Wahlbeteiligung in Bremen seit 1983 sowie der Information, dass dies ein
bundesweites Problem ist, das in Bremen besonders stark ausfällt. Die
Problematisierung der geringen Wahlbeteiligung wird mit der Frage eingeleitet,
ob und wenn ja warum aus der Sicht der SuS die geringe Wahlbeteiligung
überhaupt ein Problem darstellt (\textbf{Leitfrage}). Die Frage knüpft dabei an
die vorhergehende Stunde, in der die SuS sich mit der grundlegenden Bedeutung
von Wahlen für Demokratie beschäftigt haben. Die Antworten werden
stichpunktartig an der Tafel gesichert und führen hin zur Fokussierung des
Begriffs der allgemeinen \emph{Repräsentationsquote}. Die Bedeutung des
Begriffs wird durch die Einblendung einer Grafik visualiert.

Daraus entsteht die Leitfrage der \textbf{Erarbeitungsphase}: wer sind
eigentlich die Nichtwähler? Kann man sie bestimmten Sozialindikatoren zuordnen?
Hier nennen die SuS zunächst denkbare Indikatoren, von denen schließlich von
mir ausgewählte (die Auswahl folgt aus dem vorhandenen statistischen Material)
von den SuS in Gruppen bearbeitet werden sollen. Die Zusammenstellung der
Gruppen wird von mir vorab geplant und erfolgt leistungsheterogen, um zu
gewährleisten, dass alle Gruppenergebnisse qualitativ vergleichbar sind. Jede
Gruppe erhält ein Arbeitsblatt, in denen statistisch jeweils ein Indikator
(Arbeitslosigkeit, Wohnraum, Milieu, Bildung, Alter und Geschlecht) %falsch! Alter bezieht sich nicht auf die Ortsteile!
in Bezug
zur Wahlbeteiligung in den zehn Ortsteilen mit der höchsten und den zehn
Ortsteilen mit der niedrigsten Wahlbeteiligung gesetzt wird. Aufgabe für die
SuS ist es, anhand der gegebenen statischen Zahlen den Zusammenhang zwischen
Wahlbeteilung und Sozialstruktur zu analysieren und in maximal zwei Sätzen zu
formulieren. % Hilfekarten?
Der zuvor selbstbestimmte Schriftführer der Gruppe notiert das Ergebnis in einem vorbereiteten Padlet.\footnote{Den SuS ist
das Online Tool Padlet (\url{https://padlet.com}) aus früheren Stunden bekannt
und dient hier als interaktive digitale Tafel, die ein hohes Maß an
Interaktion, Schüleraktivität und nachhaltiger Ergebnissicherung bietet. Die
Notierung erfolgt per Smartphone, das im Politikunterricht als Recherche- und
Kollaborationstool für unterrichtliche Zwecke genutzt wird.} Anschließend
stellt jede Gruppe kurz erläuternd ihr Ergebnis vor, so dass schließlich die
soziale Selektivität als Gesamtbild erarbeitet, deutlich visualiert und
formuliert im Padlet am Smartboard steht (\textbf{Austausch- und
Sicherungsphase}). Dem Gesamtergebnis folgt die Notierung des Fachbegriffs der
\emph{sozialen Selektivität}, der das erarbeitete Ergebnis der SuS bezeichnet.

Damit ist die Leitfrage der Erarbeitungsphase beantwortet und festgehalten. In
der anschließenden \textbf{Transfer- und Diskussionsphase} wird zurückgegriffen
auf die im Einstieg nur prinzipiell problematisierte geringe
Repräsentationsquote des Wahlergebnisses, der Bürgerschaft und des Senats. Hier
wird die \textbf{Problematisierung} nun -- initiiert durch eine entsprechende
Eingangsfrage sowie einer Visualisierung durch die Verbindung der Begriffe
\emph{Repräsentationsquote} und \emph{soziale Selektivität} an Tafel und/oder
Smartboard -- vertieft und zusammengeführt mit dem Aspekt der sozialen
Selektiviät. Wahlbeteiligung wird als sozialstrukturelles Problem beurteilt und
diskutiert. Auf dieser Grundlage sollen die SuS in einer Hausaufgabe
Lösungsmöglichkeiten entwerfen und ggf. recherchieren (Vertiefung der
politischen  Urteils- und Handlungsfähigkeit).

Als \textbf{didaktische Reserve} dient die Diskussion der in der Debatte um die
niedrige Wahlbeteiligung vorgeschlagene Einführung einer Wahlpflicht \cite<vgl.
z.B.>[S. 207ff.]{kaeding2017, neu2017, schafer2015}.\footnote{Auch der damalige
Präsident der Bremischen Bürgerschaft, Christian Weber, sprach sich 2017 für
eine Wahlpflicht aus \cite{weber2017}.}
